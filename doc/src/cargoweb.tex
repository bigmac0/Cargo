%!TEX root=cargo.tex
\section{CargoWeb}

\subsection{Prerequisites}
CargoWeb has been tested with \code{node v11.7.0}, \code{npm 6.5.0},
and \code{g++ 7.3.1}.

\subsection{Getting Started}
\hi{Configuration.}
CargoWeb is completing self-contained. To get started, navigate to
\code{www} and type \code{npm install} \todo{verify the command}.
The command will automatically download the dependencies into
\code{www/node\_modules}. Then, open \code{cargo.js} and
\textbf{edit the following variables} to point to your local directories.
\begin{verbatim}
const PROB_PATH      = "/home/jpan/devel/Cargo_benchmark/problem/"
const ROAD_PATH      = "/home/jpan/devel/Cargo_benchmark/road/"
const LIBCARGO_PATH  = "/home/jpan/devel/Cargo/lib"
const LIBMETIS_PATH  = "/usr/local/lib"
const INCLUDE_PATH   = "/home/jpan/devel/Cargo/include"
\end{verbatim}

\hi{Start the server.}
Use \code{node} to launch the server.
\begin{verbatim}
> node cargo.js
Cargo listening on port 3000!
\end{verbatim}

\hi{Code Editor.}
Launch a browser and navigate to \code{http://localhost:3000}. The
CargoWeb interface will appear \todo{add screenshot}. From the
Control Pane, click on the \textit{rsalgorithm} item to toggle the dropdown.
Click on \textit{show/hide editor}. The left panel of the code editor
contains implementation for methods in an \code{RSAlgorithm}. The
right panel is the definitions file. For now, the class must be
named \code{CargoWeb}.

Some examples are provided. From the dropdown, select \code{nearest-neighbor}.
After the code appears in the code editor, hide the editor by clicking
\textit{show/hide editor} from the Control Pane.

\hi{Simulation.}
Prepare the simulation by toggling \textit{simulation settings}. \textbf{A
corresponding problem instance and road network must exist in \code{PROB\_PATH}
and \code{ROAD\_PATH}.} For the road network, three files must exist in \code{ROAD\_PATH}:
\begin{verbatim}
    {road}.rnet
    {road}.gtree
    {road}.edges
\end{verbatim}
Three additional files must exist in \code{www/public/js}:
\begin{verbatim}
    {road}_nodes.js
    {road}_positions.js
    {road}_weights.js
\end{verbatim}
Replace \code{\{road\}} with the value in the \code{road} dropdown (\code{bj5},
\code{mny}, \code{cd1}).

Then, for the select combination of the other parameters, the problem instance
with the following filename must exist in \code{PROB\_PATH}:
\begin{verbatim}
    rs-{road}-m{m}-c{c}-d{d}-s{s}-x{x}.instance
\end{verbatim}
Replace \code{m} with the value in \textit{vehicles}; \code{c} with the value
in \textit{vehicle capacity}; \code{d} with the value in \textit{delay tolerance},
not including "min"; \code{s} with the value in \textit{vehicle speed}, not including
"m/sec"; \code{x} with the value in \textit{customers scale}, not including
the text in parentheses. For example, for the values "mny", "5k", "3", "10
m/sec", "1.0x (~9.7/sec)", "6 min" in the Control Pane, the corresponding
problem instance is
\begin{verbatim}
    rs-mny-m5k-c3-d6-s10-x1.0.instance
\end{verbatim}

To start the simulation click \textit{compile + run}.

The simulation will proceed in real-time. Use the scrollwheel to zoom.  A green
spot is a moving vehicle. Grey spots are idle vehicles.  Pink squares are
customer requests.

\todo{Add screenshot}

\subsection{Under the hood}
When the \textit{compile} button is clicked,
\code{public/index.js} builds a string of the definition and implementation
and sends it via WebSocket to \code{cargo.js}. There, the string is
saved onto disk as \code{cargoweb.cc} and the server spawns \code{g++} to
compile the code. When \textit{compile + run} is clicked, the server
executes the compiled binary and begins to read its \code{stdout} and
the generated \code{cargoweb.dat} file. The contents of these two
streams are used to update the client visuals.

