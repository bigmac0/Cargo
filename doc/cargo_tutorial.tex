%!TEX root=cargo.tex
\section{Tutorial}

A good way to learn how to use the library is to study examples. In this
tutorial, I will walk you through the \code{NearestNeighbor} example found in
\code{example/nearest\_neighbor}.

First we will take a look at \code{nearest\_neighbor.h} file, reproduced below:
\begin{verbatim}
01 #include "libcargo.h"
02
03 using namespace cargo;
04
05 typedef std::tuple<DistDbl, MutableVehicleSptr> rank_cand;
06
07 class NearestNeighbor : public RSAlgorithm {
08  public:
09   NearestNeighbor();
11
12   /* My overrides */
13   virtual void handle_customer(const Customer &);
14   virtual void handle_vehicle(const Vehicle &);
15   virtual void end();
16   virtual void listen(bool skip_assigned = true, bool skip_delayed = true);
17
18  private:
19   Grid grid_;
20
21   /* Workspace variables */
22   vec_t<Stop> sch;
23   vec_t<Wayp> rte;
24   MutableVehicleSptr best_vehl;
25   vec_t<MutableVehicleSptr> candidates;
26   bool matched;
27   tick_t timeout_0;
28
29   void reset_workspace();
30 };
\end{verbatim}

The library is included in line 01, then the namespace is used in line 03 to
avoid appending the \code{cargo} prefix. Nearest neighbor depends on ranking
vehicles by proximity to customer, so on line 05 a type is defined to attach
a rank to each vehicle. The rank is \code{DistDbl} type, defined in
\code{include/types.h}. Then, the vehicle is kept as a pointer
\code{MutableVehicleSptr}; this type is defined in \code{include/classes.h}.

Line 07 begins the \code{NearestNeighbor} class definition; it inherits from
\code{RSAlgorithm} (\code{include/rsalgorithm.h}) in order to access the
basic listening and matching functionality.

Line 08 opens the public members. Line 09 declares the constructor. Lines
12--16 declare the base functionality overrides that we will provide. The
method signatures MUST match the base signatures in \code{rsalgorithm.h}.

Line 16 declares the \code{listen} method. This method takes two paramters.
The first parameter forces the method to not include customers that are already
assigned, but not yet picked up, from the customers queue. The second
parameter makes the method skip those customers that are not matched but have
already been attempted within the past \code{retry} period. Set either to
\code{false} to disable the functionality.



